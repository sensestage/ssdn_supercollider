\documentclass[letterpaper,10pt]{article}

\usepackage{rotating}
\usepackage{url}

%opening
\title{Sense World Data Network}
\author{Marije Baalman}

\begin{document}

% \textwidth 17cm
% \hoffset 1cm
% \voffset 1cm

\maketitle

\begin{abstract}
The data network framework is meant to make sharing of data (from sensors or internal processes) between collaborators in an interactive media art work easier, faster and more flexible. There is a central host, which receives all data, and manages the client connections. Each client can subscribe to data \textit{nodes}, to use that data in its own internal processes; and each client can publish data onto the network, by creating a node. A new client can query the network which nodes are present and is informed when new nodes appear after the client has been registered.
\end{abstract}

\section{Data Network Elements}

The data network is built up from different elements:
\begin{description}
 \item [DataNetwork] the network itself
 \item [DataNode] a node is a collection of slots, usually based upon a device or another common source (e.g. result from a function).
 \item [DataSlot] a slot is a single data stream
\end{description}

Data on the network is set by calling the function method \verb|setData| with as arguments the node ID and an array of data values (either numbers (floats) or strings). The ID is an unique identifier (an integer). The function can be called for example by a class instance that parses serial data.

Each \textbf{DataNode} and each \textbf{DataSlot} can be given a label, so that their functionality becomes more human understandable.

\section{OSC interface}
There is an OSC interface to the network, which allows clients to become part of the data network and access its data, and also create its own data nodes on the network.

The network will announce itself to the broadcast address of the network, to a number of ports (default: range 6000-6009, and 57120-57129), so that clients can automatically configure to connect to the network, as soon as it is in the air.

A textfile with the network's OSC port can be found in the file \url{http://hostip/SenseWorldDataNetwork}\footnote{e.g. for a host with IP 192.168.1.7 the url is: \url{http://192.168.1.7/SenseWorldDataNetwork}}, which can be retrieved by clients, so they know where to send the registration message.

The general setup is that an OSC client first sends a register message to the data network server. Then it will start receiving ping messages, to which it has to reply with pong messages. The client has to query which nodes and slots are present on the network after registering, so it will receive info messages on each node and slot. Then it can subscribe to nodes and slots, and will receive data from the nodes and slots it is subscribed to via the data messages.

The client can supply a new node to the network, by using the /set/data message; it can also label the nodes and slots thus created.
Whenever a new node or slot is added (or changed, e.g. when it gets a label), the client will receive a new info message.
If there occurs an error in the communication, then an error message is sent. The unregister message only needs to be sent, if for example the client crashed and is trying to reconnect on the same port.

All messages to the server now have a reply, which is either the requested info, a confirmation message, or a warning or error.

See table \ref{oscinterface} for an overview of commands.

% \begin{table}
\begin{sidewaystable}[!tbp]
\small
\begin{center}
% use packages: array
\begin{tabular}{|llll|}
\hline
\verb|/datanetwork/announce| & si & host, port no. & announce the network with its coordinates \\
\verb|/datanetwork/quit| & si & host, port no. & inform that the host has quit \\

\verb|/register| & is & port no., name & register to the network as a client, the name is used as an identifier for the client to remember settings \\
\verb|/registered| & i & port no. & reply to register to the network as a client \\

\verb|/unregister| & i & port no. & unregister to the network as a client \\  
\verb|/unregistered| & i & port no. & reply to unregister to the network as a client \\

\verb|/ping| & i & port no. &  message to check if client is still there \\
\verb|/pong| & i & port no. &  expected reply to the \verb|/ping| message \\

\verb|/error| & ssi & cause, error message, error ID & error occurred upon request (indicated by cause) \\
\verb|/warn| & ssi & cause, warn message, error ID & non fatal error occurred upon request \\  

\verb|/query/expected| & i & port no.& query which nodes are expected in the network (reply /info/expected) \\
\verb|/query/nodes| & i & port no.& query which nodes are in the network (reply /info/node) \\
\verb|/query/slots| & i & port no.& query which slots are in the network (reply /info/slot)\\
\verb|/query/clients| & i & port no.& query which clients are in the network (reply /info/client)\\
\verb|/query/setters| & i & port no.& query which nodes the client is the setter of (reply /info/setter) \\
\verb|/query/subscriptions| & i & port no.& query which subscriptions the client has (reply /subscribed/node, /subscribed/slot)\\

\verb|/info/expected| & i(s) & node ID, node label & info about an expected node \\
\verb|/info/node| & isii & node ID, node label, number of slots, node type & info about a node \\
\verb|/info/slot| & iisi & node ID, slot ID, slot label, slot type & info about a slot \\
\verb|/info/client| & sis & ip, port no., hostname & info about a client \\
\verb|/info/setter| & isi & node ID, node label, number of slots & info about a node the client is setting \\

\verb|/subscribe/all| & i & port no. & subscribe to receive data from all nodes \\
\verb|/unsubscribe/all| & i & port no. & unsubscribe from all nodes \\

\verb|/subscribe/node| & ii & port no., node ID & subscribe to receive data from a node \\
\verb|/subscribed/node| & ii & port no., node ID & reply to subscribe to receive data from a node \\

\verb|/unsubscribe/node| & ii & port no., node ID & unsubscribe to receive data from a node \\
\verb|/unsubscribed/node| & ii & port no., node ID & reply to unsubscribe to receive data from a node \\

\verb|/subscribe/slot| & iii & port no., node ID, slot ID & subscribe to receive data from a slot \\  
\verb|/subscribed/slot| & iii & port no., node ID, slot ID & reply to subscribe to receive data from a slot \\  

\verb|/unsubscribe/slot| & iii & port no., node ID, slot ID & subscribe to receive data from a slot \\  
\verb|/unsubscribed/slot| & iii & port no., node ID, slot ID & reply to unsubscribe to receive data from a slot \\  

\verb|/data/node| & iff..f & node ID, data values & node data \\
\verb|/data/node| & iss..s & node ID, string data values & node data \\
\verb|/get/node| & ii & port no., node ID & get data from a node (reply /data/node) \\

\verb|/data/slot| & iif & node ID, slot ID, data value & slot data \\
\verb|/data/slot| & iis & node ID, slot ID, string data value & slot data \\
\verb|/get/slot| & iii & port no., node ID, slot ID & get data from a slot (reply /data/slot) \\

\verb|/set/data| & iif..f & port no., node ID, data values & set data to a node (reply /data/node)\\
\verb|/set/data| & iis..s & port no., node ID, string data values & set data to a node\\

\verb|/label/node| & iis & port no., node ID, node label & set label to a node \\
\verb|/label/slot| & iiis & port no., node ID, slot ID, slot label & set label to a slot \\

\verb|/remove/node| & ii & port no., node ID & remove a node (only possible if client is setter) \\
\verb|/removed/node| & i & node ID & reply to remove a node \\
\verb|/remove/all| & i & port no. & remove all nodes the client is a setter of (generates /removed/node messages) \\

\verb|/add/expected| & ii(isi) & port no., node ID, node size, node label, node type & add an expected node to the network (reply /info/expected) \\
 &  &  & if node size is given, the node is created as well (and generates a /info/node message) \\
 &  &  & node type is 0: float, 1: string (default is 0) \\
\hline
\end{tabular}
\end{center}
\caption{OSC namespace for the Data Network}
\label{oscinterface}
% \end{table}
\end{sidewaystable}

\begin{table}
\small
\begin{center}
\begin{tabular}{|rl|}
 \hline
  1 & "Client with IP"+addr.ip+"and port"+addr.port+"is not registered. Please register first" \\
  2 & "Client with IP"+addr.ip+"and port"+addr.port+"is already registered. Please unregister first" \\
  3 & "Client with IP"+addr.ip+"and port"+addr.port+"was not registered"  \\
  4 & "Client with IP"+addr.ip+"and port"+addr.port+"is not the setter of node with id"+..  \\
  5 & "Node with id"+..+"is not part of the network" \\
  6 & "Node with id"+..+"is not expected to be part of the network"  \\
  7 & "There are no expected nodes in the network" \\
  8 & "There are no nodes in the network"  \\
  9 & "There are no clients in the network"  \\
 10 & "Client with IP"+addr.ip+"and port"+addr.port+"has no setters"  \\
 11 & "Client with IP"+addr.ip+"and port"+addr.port+"has no subscriptions" \\
 12 & "Node with id"+...+"does not have"+..+"slots" \\
 12 & "Node with id"+...+"does not have"+..+"slots" \\
 13 & "Node with id"+...+"has wrong type"+... \\
 \hline
 \end{tabular}
\end{center}
\caption{Error codes and strings}
\label{errorcodes}
\end{table}


\section{Max implementation (by Harry Smoak, Joseph Malloch and Brett Bergmann)}
In the Max implementation, there is a data \textit{sink}, which manages the connection to the network (registering, subscriptions, etc.), and gives the received data. There is a data \textit{source}, which can send data into the network. The subscriptions are handled by textfiles, as are the published data nodes, so they can be easily restored upon opening a max patch.
The objects react to the announce message from the network to set the right host IP and port.

The Max patch dn.node can receive data from any number of nodes, as it takes multiple arguments (i.e. the object "dn.node 10 12 15" would subscribe to data from nodes 10, 12 and 15)

\section{SuperCollider implementation}

The SuperCollider implementation is done in a set of classes.

Documentation for these is available in HTML format.

\section{Installation}

\subsection{SuperCollider Quark}

The DataNetwork can be most easily installed from SuperCollider's Quarks extension management system.
This also includes the client patches for other software environments.

To install the classes, do the following inside SuperCollider:

\begin{verbatim}

Within SuperCollider do the following:
// check out all quarks:
Quarks.checkoutAll;

// or update them all to the latest version:
Quarks.update;


// install the SenseWorld DataNetwork quark
// - this will install all other quarks that are needed
Quarks.install("SenseWorld DataNetwork");

// recompile the library 

//---------------- host -----------------------

// On OSX, add this to the startup file (or execute it each time)
SWDataNetworkOSC.httppath = "/Library/WebServer/Documents/";


// create a network:
x = SWDataNetwork.new;
// add the OSC interface
x.addOSCInterface;

//---------------- client ----------------------

// create a network client:
y = SWDataNetworkClient.new( ~hostip, "myname" );



// For more help, access the helpfile:
SenseWorldDataNetwork
\end{verbatim}


\subsection{Apache}

You need to install a webserver such as Apache on the host system.

(package apache2 on Debian/Ubuntu; usually available on OSX)

The general files will be put in \verb|/var/www|.
You have to make this directory writable by the user by executing (as root)

\begin{verbatim}
 cd /var/www
 chmod 775 .
 chgrp netdev .
\end{verbatim}

Assuming that the user running SuperCollider is member of the group \verb|netdev|. You can check this by:

\begin{verbatim}
 groups
\end{verbatim}

To add yourself to the group, execute as root (with instead of ``nescivi'' your username):
\begin{verbatim}
 adduser nescivi netdev
\end{verbatim}

You may need to logout and log back in for this to take effect.

On OSX the default path for http files is: \verb|/Library/WebServer/Documents/|. You may have to enable ``Web Sharing'' in the ``Sharing'' settings in the System Preferences.

% \subsection*{TODO (SuperCollider)}

% \begin{itemize}
%  \item (discard) change expectedNodes to become a security option
% \end{itemize}

\section*{Acknowledgments}\label{sec:acknowledgments}
This software was created by

\textbf{SuperCollider classes:}
Marije Baalman (nescivi)  

\textbf{Max patches:}
Harry Smoak, 
Joseph Malloch, 
Brett Bergmann


Developed as part of the "Sense/Stage" project and the "Papyrus" project between\\
Design and Computation Arts, Fine Arts, Concordia University\\
and\\
Input Devices and Music Interaction Lab, Music Technology, McGill University

This work was support by grants from the Social Sciences and Humanities Research Council of Canada and the Hexagram Institute for Research/Creation in Media Arts and Sciences, Montr\'eal, QC, Canada.

(c) 2008-9 by the authors

Released under the GNU/GPL (see COPYING file)

\subsection*{ChangeLog}

\begin{itemize}
 \item ------------- v 0.3 --------------
 \item 9/7/2009 - improved data logging and playback support. Added string data nodes (adds a type tag to some osc messages). Logging now saves the spec with labels.
 \item ------------- v 0.2 --------------
 \item 27/5/2009 - added remove all message. error codes for error and warn messages.
 \item 19/5/2009 - performance improvement, fix bugs in gui, logging option for osc communication, logging option for update times, clients are now removed after a certain amount of missed pongs. Protection for non-numerical data coming in.
 \item 18/5/2009 - added subscribe all and unsubscribe all messages.
 \item 4/4/2009 - added client gui, and updated the client in SC.
 \item 2/4/2009 - added gui for connected clients
 \item 1/4/2009 - added help files and wii mote support, improved main gui
 \item 12/3/2009 - added pattern support
 \item 12/3/2009 - create a bridge from GeneralHID, including some other bugfixes
 \item 12/3/2009 - added a size argument to expected nodes; if set, this will create the node already with the given size, with data values 0, so that properties of the node and slots can be set. (to fix the todo: create ``virtual nodes'' for nodes that are expected but not there yet, so some settings can already be set)
 \item 12/3/2009 - implemented the port storage in a file mechanism
 \item 21/11/2008 - implemented backup mechanism for reconnection of any clients that were connected before a restart and the SC client version
 \item 21/11/2008 - added warn message for some actions
 \item 06/10/2008 - added announce message
 \item 06/10/2008 - added acknowledgement messages for actions that do not have an immediate reply otherwise
 \item 06/10/2008 - changed so that nodeID's and slotID's now are always integers.
\end{itemize}



\end{document}
